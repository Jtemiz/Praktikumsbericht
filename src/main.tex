%% Dokumenteinstellungen
\documentclass[
type=Projektarbeit,
status=draft, % Ändern zu "final" für die Abgabeversion
language=german, % oder: "english"
bibengine=bibtex,
]{unibwm-inf-thesis}

\usepackage{textcomp}
\usepackage{verbatim}
\usepackage{listings}
\bibliographystyle{apalike-german}

\newcommand{\todo}[1]{\textbf{TODO: #1}}
\setvalue{title}{Konzeptentwicklung zur Umsetzung von Body-Tracking bei Kindern}
\setvalue{author2}{Joel Temiz}
\setvalue{matrnr2}{1196670}
\setvalue{author1}{Dustin Hoffmann}
\setvalue{matrnr1}{1196346}

\setvalue{examiner}{Prof. Dr. Ulrike Lechner}
\setvalue{advisor}{Lisa Verlande}
\setvalue{advisorlabel}{Betreuerin}
\setvalue{seminartitle}{Praxisprojekt}
\setvalue{seminarcontext}{Sommertrimester 2022}
\setvalue{deadline}{30.06.2022}
\usepackage{amsmath}
\usepackage{nameref}

\begin{document}
\tableofcontents

\mainmatter
\begin{abstract}
    abstract
\end{abstract}
\chapter{Projekttagebuch}\label{ch:projekttagebuch}
DPP -> Digitales Prüf- \& Produktionskonzept
\section*{Woche 1: 04.07. - 08.07.}
Montag - Mittwoch: Klausur\\
Donnerstag: Besprechung Vorhaben, Zielsetzung, Ideensammlung\\
Besprechung des Vorhabens: Anfängliche Ordnung der Ideen in MindMap.  \dots\\
Ergebnis: Fokus des Praktikums: Computer Vision (Visuelle Überprüfung von Schaltschränken durch Kamerasystem)\\
Betrachten Gegenstück bei DMG (Radmagazin-Prüfstand) \\
Freitag: Anfängliche Machbarkeitsanalyse --> Ist es möglich, mit einer Tiefenkamera (\zB Intel RealSense) existierende Schaltschränke mit CAD-Zeichung zu vergleichen=

\section*{Woche 2: 11.07. - 15.07.}
\section*{Woche 3: 18.07. - 22.07.}
\section*{Woche 4: 25.07. - 29.07.}
\section*{Woche 5: 01.08. - 05.08.}
\section*{Woche 6: 08.07. - 12.08.}
\section*{Woche 7: 15.07. - 19.08.}
\section*{Woche 8: 22.07. - 26.08.}
\section*{Woche 9: 29.07. - 02.09.}



\bibliography{main}

\newpage

\thispagestyle{empty}

\begin{large}

\vspace*{2cm}

\noindent
Hiermit versichern wir, die vorliegende Arbeit selbständig und ohne fremde Hilfe verfasst, die Zitate ordnungsgemäß gekennzeichnet und keine anderen, als die im Literatur/Schriftenverzeichnis angegebenen Quellen und Hilfsmittel benutzt zu haben.\\[1em]

\noindent


\vspace{2cm}

\noindent
Neubiberg, den 30. Juni 2022

\vspace{3cm}

\dotfill
\hspace*{3cm}%
\dotfill \\
\hspace*{0.3cm}
\textit{(Unterschrift des 1. Kandidaten)}
\hspace*{3.2cm}%
\textit{(Unterschrift des 2. Kandidaten)}

\end{large}


\end{document}
