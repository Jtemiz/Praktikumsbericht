%% Dokumenteinstellungen
\documentclass[
    type=Projektarbeit,
    status=draft, % Ändern zu "final" für die Abgabeversion
    language=german, % oder: "english"
    bibengine=bibtex,
]{unibwm-inf-thesis}

\usepackage{textcomp}
\usepackage{verbatim}
\usepackage{listings}
\bibliographystyle{apalike-german}

\newcommand{\todo}[1]{\textbf{TODO: #1}}
\setvalue{title}{Konzeptentwicklung zur Umsetzung von Body-Tracking bei Kindern}
\setvalue{author2}{Joel Temiz}
\setvalue{matrnr2}{1196670}
\setvalue{author1}{Dustin Hoffmann}
\setvalue{matrnr1}{1196346}

\setvalue{examiner}{Prof. Dr. Ulrike Lechner}
\setvalue{advisor}{Lisa Verlande}
\setvalue{advisorlabel}{Betreuerin}
\setvalue{seminartitle}{Praxisprojekt}
\setvalue{seminarcontext}{Sommertrimester 2022}
\setvalue{deadline}{30.06.2022}
\usepackage{amsmath}
\usepackage{nameref}

\begin{document}
    \tableofcontents

    \mainmatter
    \begin{abstract}
        abstract
    \end{abstract}


    \chapter{Einleitung und Motivation}


    \chapter{Gesamtkonzept von DPP und Einordnung in Fertigungsprozess}


    \chapter{Implementierung CV-Modul}
    Teilmodul von AS DPP: DPP Visual\\
    Ziel von DPP Visual: Sind alle notwendigen Bauteile vorhanden?
    Sind die Bauteile an der richtigen Stelle verbaut?\\
    Ansatz zur technischen Umsetzung: Anlernen durch Fotos der Bauteile.
    Abgleich der Bilder des Schaltschranks mit Bildern der Bauteile --> Generierung von Keys mit Beschreibung, die
    Übereinstimmung beschreiben.\\
    Problem: Übereinstimmungen teilweise fehlerhaft \\
    Lösung: Clustering der Positionen der Matches mittels KMeans-Algorithmus.
    Vergleich der Ähnlichkeit der Farbhistogramme von Template und Cluster


    \section{Programmiersprache}
    Im Zuge der Implementierung des Moduls DPP Visual werden zahlreiche Methoden der Bildverarbeitung und des
    Computer-Visions benötigt.
    Zudem werden Algorithmen zum Umgang mit großen Datenmengen, wie Clustering verwendet.
    Eine weitere Anforderung für den weiteren Verlauf ist die Unterstützung der Intel RealSense SDK, die benötigt wird,
    um Aufnahmen der 3D-Kamera zu verarbeiten.

    Eine Programmiersprache die diese Anforderungen allesamt erfüllt ist Python.
    Diese bietet neben der Unterstützung der Intel RealSense SDK zahlreiche Libraries wie OpenCV, Numpy, Pandas,
    Pillow etc., die bei der Lösung von Problemen in der Mathematik sowie der Computer-Vision unterstützen.

    Diese Libraries können mit vergleichsweise geringem Aufwand über den Pip-Installer in die Anwendung integriert
    werden, sodass der Funktionsumfang deutlich erweitert werden kann.

    Für die Umsetzung der Testumgebung wurde zusätzlich das pythonbasierte Framework \textit{Flask} mit der
    Template-Engine \textit{Jinja} herangezogen.
    Diese Kombination ermöglicht es eine webbasierte Oberfläche umzusetzen, die das Testen mit einem mobilen Endgerät
    (Tablet, Smartphone) ermöglicht und somit die simultane Überprüfung der Anwendungsergebnisse vereinfacht.


    \section{Konzeption DPP Visual}


    \chapter{Testumgebung CV-Modul}



    \bibliography{main}

    \newpage

\thispagestyle{empty}

\begin{large}

\vspace*{2cm}

\noindent
Hiermit versichern wir, die vorliegende Arbeit selbständig und ohne fremde Hilfe verfasst, die Zitate ordnungsgemäß gekennzeichnet und keine anderen, als die im Literatur/Schriftenverzeichnis angegebenen Quellen und Hilfsmittel benutzt zu haben.\\[1em]

\noindent


\vspace{2cm}

\noindent
Neubiberg, den 30. Juni 2022

\vspace{3cm}

\dotfill
\hspace*{3cm}%
\dotfill \\
\hspace*{0.3cm}
\textit{(Unterschrift des 1. Kandidaten)}
\hspace*{3.2cm}%
\textit{(Unterschrift des 2. Kandidaten)}

\end{large}


\end{document}
