%% Dokumenteinstellungen
\documentclass[
    type=Projektarbeit,
    status=draft, % Ändern zu "final" für die Abgabeversion
    language=german, % oder: "english"
    bibengine=bibtex,
]{unibwm-inf-thesis}

\usepackage{textcomp}
\usepackage{verbatim}
\usepackage{listings}
\bibliographystyle{apalike-german}

\newcommand{\todo}[1]{\textbf{TODO: #1}}
\setvalue{title}{Qualitätssicherung im Schaltanlagenbau: Konzeption und Implementierung DPP}
\setvalue{author}{Joel Temiz}
\setvalue{matrnr}{1196670}

\setvalue{examiner}{Prof. Dr. Ulrike Lechner}
\setvalue{advisor}{Lisa Verlande}
\setvalue{advisorlabel}{Betreuerin}
\setvalue{seminartitle}{Praxisprojekt}
\setvalue{seminarcontext}{Sommertrimester 2022}
\setvalue{deadline}{30.06.2022}
\usepackage{amsmath}
\usepackage{nameref}

\begin{document}
    \tableofcontents

    \mainmatter
    \begin{abstract}
        abstract
    \end{abstract}


    \chapter{Einleitung und Motivation}


    \chapter{Gesamtkonzept von DPP und Einordnung in Fertigungsprozess}
    Zur Umsetzung eines betrieblichen Anwendungssystems muss im ersten Schritt eine Konzeption erfolgen, die darstellt,
    wie die Anwendung gestaltet sein kann, um das gewünschte Ziel zu erfüllen.
    \section{Anforderungen}
    In Rücksprache mit dem Praxispartner, wurden einerseits die Ziele des Systems \textit{DPP} herausgearbeitet.
    Andererseits wurden einzelne Anforderungen konkretisiert, die dazu dienen sollen das Ziel zu erfüllen.
    Das Hauptziel des Anwendungssystems \textit{DPP} ist die Minimierung von fehlerhaften Schaltanlagen.
    Als Anforderung wurden hierbei eine maximale Fehlerquote von 3\% genannt.

    Um dieser Anforderung gerecht werden zu können, stellt sich der Praxispartner verschiedene Leistungen vor.
    Diese sind einerseits dem Prüfprozess zuordenbar, andererseits soll bereits die Entstehung von Fehlern vermieden
    werden, indem die Monteure während der Produktion unterstützt werden.

    \subsection{Digitale Maschinenakte}
    \begin{itemize}
        \item Einführung einer Maschinenakte
        \item Erstellung einer Maschinenakte für jeden Schaltschrank bei Auftragseingang
        \item Identifizierbarkeit des Schaltschranks
        \item Maschinenakte enthält alle relevanten Informationen über Produktion und Wartung der Maschine.
        \item Bsp: CAD-Zeichnung, EPlan, Simulation der zugehörigen Maschine, \ldots
    \item \end{itemize}
    \subsection{Technische Prüfung}
    Die erste Leistung ist die technische Prüfung der Schaltschränke durch Simulation.
    Hierbei soll die Maschine, für die die Schaltanlage gebaut wird, durch ein System simuliert werden.
    %TODO Simulation erklären
    
    \subsection{Visuelle Prüfung}
    \begin{itemize}
        \item Hauptfokus der Arbeit
        \item Mehrere Aufgaben:
        \item Verfügbarkeit der Bauteile
        \item Richtige Position der Bauteile
        \item Optische Beschädigungen
    \end{itemize}


    \subsection{Fehlerfreies Arbeiten}
    \begin{itemize}
        \item Umsetzung derzeit noch nicht geplant
        \item Idee:
        \item Unterstützung der Monteure in der Produktion
        \item z.B. durch AR -> Brillen zeigen die Position des einzubauenden Bauteils.
        \item Cobotik --> Drehmoment-Schlüssel stellt automatisch drehmoment ein, zieht schrauben nach.
    \end{itemize}



    \chapter{Implementierung \textit{Visuelle Prüfung}}
    Teilmodul von AS DPP: DPP Visual\\
    Ziel von DPP Visual: Sind alle notwendigen Bauteile vorhanden?
    Sind die Bauteile an der richtigen Stelle verbaut?\\
    Ansatz zur technischen Umsetzung: Anlernen durch Fotos der Bauteile.
    Abgleich der Bilder des Schaltschranks mit Bildern der Bauteile --> Generierung von Keys mit Beschreibung, die
    Übereinstimmung beschreiben.\\
    Problem: Übereinstimmungen teilweise fehlerhaft \\
    Lösung: Clustering der Positionen der Matches mittels KMeans-Algorithmus.
    Vergleich der Ähnlichkeit der Farbhistogramme von Template und Cluster


    \section{Programmiersprache}
    Im Zuge der Implementierung des Moduls DPP Visual werden zahlreiche Methoden der Bildverarbeitung und des
    Computer-Visions benötigt.
    Zudem werden Algorithmen zum Umgang mit großen Datenmengen, wie Clustering verwendet.
    Eine weitere Anforderung für den weiteren Verlauf ist die Unterstützung der Intel RealSense SDK, die benötigt wird,
    um Aufnahmen der 3D-Kamera zu verarbeiten.

    Eine Programmiersprache die diese Anforderungen allesamt erfüllt ist Python.
    Diese bietet neben der Unterstützung der Intel RealSense SDK zahlreiche Libraries wie OpenCV, Numpy, Pandas,
    Pillow etc., die bei der Lösung von Problemen in der Mathematik sowie der Computer-Vision unterstützen.

    Diese Libraries können mit vergleichsweise geringem Aufwand über den Pip-Installer in die Anwendung integriert
    werden, sodass der Funktionsumfang deutlich erweitert werden kann.

    Für die Umsetzung der Testumgebung wurde zusätzlich das pythonbasierte Framework \textit{Flask} mit der
    Template-Engine \textit{Jinja} herangezogen.
    Diese Kombination ermöglicht es eine webbasierte Oberfläche umzusetzen, die das Testen mit einem mobilen Endgerät
    (Tablet, Smartphone) ermöglicht und somit die simultane Überprüfung der Anwendungsergebnisse vereinfacht.


    \section{Konzeption DPP Visual}


    \chapter{Testumgebung CV-Modul}



    \bibliography{main}
    \backmatter
    \chapter{Projekttagebuch}\label{ch:projekttagebuch}
DPP -> Digitales Prüf- \& Produktionskonzept
\section*{Woche 1: 04.07. - 08.07.}
Montag - Mittwoch: Klausur\\
Donnerstag: Besprechung Vorhaben, Zielsetzung, Ideensammlung\\
Besprechung des Vorhabens: Anfängliche Ordnung der Ideen in MindMap.  \ldots\\
Ergebnis: Fokus des Praktikums: Computer Vision (Visuelle Überprüfung von Schaltschränken durch Kamerasystem)\\
Betrachten Gegenstück bei DMG (Radmagazin-Prüfstand) \\
Freitag: Konzepterstellung: Visuelle Überprüfung von Schaltschränken: Was soll überprüft werden?
Wie kann das überprüft werden?
\begin{itemize}
    \item Alle Baugruppen vorhanden (\zB Steuerungen, Netzteile, \ldots)?
    \item Bauteile an richtiger Stelle?
    \item Kabelkanäle korrekt gesetzt?
    \item Äußere Beschädigungen des Schaltschranks (\zB Kratzer)
\end{itemize}
Erstellen von Arbeitspaketen: 1. Arbeitspaket: Erkennung von Bauteilen im Gesamtbild anhand einer oder mehrerer Vorlagen

\section*{Woche 2: 11.07. - 15.07.}
\begin{itemize}
    \item Einarbeitung in Kamerasystem (Intel RealSense D435)
    \item Aufsetzen der Programmierumgebung (Intel RealSense SDK, Connectivity zwischen IRS-SDK und Python, OpenCV, \ldots)
    \item Einarbeitung in CV-Libraries (Pillow, CV2, IRS)
    \item Überprüfung Verfügbarkeit von Third-Party-Tools
    \item Anfängliche Überlegung in Umsetzung der Objekterkennung mit verfügbaren Mitteln (begrenzte Rechenleistung)
\end{itemize}
\section*{Woche 3: 18.07. - 22.07.}
\begin{itemize}
    \item Beginn Implementierung Visual Komponente
    \item Ziel: Überprüfung der Verfügbarkeit + richtigen Position aller Bauteile
    \item Ansatz: Feature-Matching
    \item Aufsetzen Testumgebung auf Web-Basis, um Erfolg kontinuierlich zu Überprüfen
\end{itemize}
\section*{Woche 4: 25.07. - 29.07.}
    \begin{itemize}
        \item {Montag, Dienstag Homeoffice}
        \item Weiterentwicklung: Hinzunahme Farbhistogramm
    \end{itemize}
\section*{Woche 5: 01.08. - 05.08.}
\section*{Woche 6: 08.07. - 12.08.}
\section*{Woche 7: 15.07. - 19.08.}
\section*{Woche 8: 22.07. - 26.08.}
\section*{Woche 9: 29.07. - 02.09.}

    \newpage

\thispagestyle{empty}

\begin{large}

\vspace*{2cm}

\noindent
Hiermit versichern wir, die vorliegende Arbeit selbständig und ohne fremde Hilfe verfasst, die Zitate ordnungsgemäß gekennzeichnet und keine anderen, als die im Literatur/Schriftenverzeichnis angegebenen Quellen und Hilfsmittel benutzt zu haben.\\[1em]

\noindent


\vspace{2cm}

\noindent
Neubiberg, den 30. Juni 2022

\vspace{3cm}

\dotfill
\hspace*{3cm}%
\dotfill \\
\hspace*{0.3cm}
\textit{(Unterschrift des 1. Kandidaten)}
\hspace*{3.2cm}%
\textit{(Unterschrift des 2. Kandidaten)}

\end{large}


\end{document}
