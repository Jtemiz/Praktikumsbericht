\chapter{Projekttagebuch}\label{ch:projekttagebuch}
DPP -> Digitales Prüf- \& Produktionskonzept
\section*{Woche 1: 04.07. - 08.07.}
Montag - Mittwoch: Klausur\\
Donnerstag: Besprechung Vorhaben, Zielsetzung, Ideensammlung\\
Besprechung des Vorhabens: Anfängliche Ordnung der Ideen in MindMap.  \ldots\\
Ergebnis: Fokus des Praktikums: Computer Vision (Visuelle Überprüfung von Schaltschränken durch Kamerasystem)\\
Betrachten Gegenstück bei DMG (Radmagazin-Prüfstand) \\
Freitag: Konzepterstellung: Visuelle Überprüfung von Schaltschränken: Was soll überprüft werden?
Wie kann das überprüft werden?
\begin{itemize}
    \item Alle Baugruppen vorhanden (\zB Steuerungen, Netzteile, \ldots)?
    \item Bauteile an richtiger Stelle?
    \item Kabelkanäle korrekt gesetzt?
    \item Äußere Beschädigungen des Schaltschranks (\zB Kratzer)
\end{itemize}
Erstellen von Arbeitspaketen: 1. Arbeitspaket: Erkennung von Bauteilen im Gesamtbild anhand einer oder mehrerer Vorlagen

\section*{Woche 2: 11.07. - 15.07.}
\begin{itemize}
    \item Einarbeitung in Kamerasystem (Intel RealSense D435)
    \item Aufsetzen der Programmierumgebung (Intel RealSense SDK, Connectivity zwischen IRS-SDK und Python, OpenCV, \ldots)
    \item Einarbeitung in CV-Libraries (Pillow, CV2, IRS)
    \item Überprüfung Verfügbarkeit von Third-Party-Tools
    \item Anfängliche Überlegung in Umsetzung der Objekterkennung mit verfügbaren Mitteln (begrenzte Rechenleistung)
\end{itemize}
\section*{Woche 3: 18.07. - 22.07.}
\begin{itemize}
    \item Beginn Implementierung Visual Komponente
    \item Ziel: Überprüfung der Verfügbarkeit + richtigen Position aller Bauteile
    \item Ansatz: Feature-Matching
    \item Aufsetzen Testumgebung auf Web-Basis, um Erfolg kontinuierlich zu Überprüfen
\end{itemize}
\section*{Woche 4: 25.07. - 29.07.}
    \begin{itemize}
        \item {Montag, Dienstag Homeoffice}
        \item Weiterentwicklung: Hinzunahme Farbhistogramm
        \item Vorbereitung Zwischenpräsentation
        \item Weiterentwicklung: Umsetzung ML-basierte Objekterkennung
    \end{itemize}
\section*{Woche 5: 01.08. - 05.08.}
    \begin{itemize}
        \item Weiterentwicklung ML-basierte Objekterkennung
        \item Training Positiver und negativer Bilder anhand Haar Cascade Algorithmus
        \item Ergebnis: Muss optimiert werden
    \end{itemize}
\section*{Woche 6: 08.07. - 12.08.}
\section*{Woche 7: 15.07. - 19.08.}
\section*{Woche 8: 22.07. - 26.08.}
\section*{Woche 9: 29.07. - 02.09.}
